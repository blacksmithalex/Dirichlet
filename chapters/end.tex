\chapter{Заключение}

Подводя итог всему выше написанному, я хочу заметить, что Дирихле сделал огромный вклад в науку не только как ученый и математик, но и как преподаватель. Последователями Дирихле стал целый ряд учёных. Среди них такие известные немецкие математики, как Фердинанд Эйзенштейн, Леопольд Кронекер, Рудольф Липшиц и многие другие. Многочисленность учеников и их плодотворная научная деятельность наглядно доказывает, что труды Лежёна Дирихле действительно были очень значимыми и внесли огромный вклад в науку Германии.

Дирихле прожил не самую долгую жизнь: он скончался в возрасте пятидесяти четырех лет. Его столь ранний уход из жизни был связан с тем, что он всю свою жизнь посвятил науке, при этом не отдавая должного внимания своему здоровью. Болезни дали о себе знать и стали причиной его смерти. 

Меня всегда впечатляли такие личности, ведь открытия, которые делают и делали ученые напрямую связаны с личностными качествами. Составляя этот реферат, я освежил часть знаний, которые были усвоены мной на начальных курсах механико-математического факультета МГУ, а также узнал много интересного об этом ученом. Надеюсь, что читатель также найдет этот реферат полезным для себя, и, самое главное, вдохновиться на дальнейшее изучение математики.



