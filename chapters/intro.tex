\chapter{Введенеие}

Тема моего реферата, как уже можно было понять из названия - книга П.Г. Лежена Дирихле ”Лекции по теории чисел”. По моему мнению, теория чисел - это один из основополагающих разделов математики. Зародившись еще в древнем мире, этот раздел математики смог сохранить свою целостность и сильно спрогрессировать за последние столетия. По своим методам теория чисел делится на четыре части: элементарную, аналитическую, алгебраическую и геометрическую. Сейчас методы теории чисел широко применяются в криптографии, вычислительной математике, информатике. 

Лично для меня, теория чисел - это тот раздел, который сочетает в себе элегантность формулировок, красоту свойств и изящные ходы в доказательствах. Помню, как на первом курсе я слушал лекции Н.Г. Мощевитина и не раз удивлялся тому, как даже из понятных обычному человеку вещей можно за полтора часа вывести фундаментальный теоритический базис и сформулировать ряд проблем, которые до сих пор не нашли решение. Особенно мне запомнилась проблема простых чисел-близнецов, так как утверждения, связанные с простыми числами у меня всегда вызывали неподдельный интерес.

Также отмечу, что и сейчас я часто сталкиваюсь с вопросами из области теории чисел, решая уже вычислительные задачи по олимпиадному программированию и не только. Понимание основ теории чисел помогает продумывать алгоритмы с более оптимальной асимптотической сложностью, а также видеть более быстрые ходы там, где обычный человек, интересующийся математикой, не заметил бы. 

Надеюсь я смог заинтересовать читателя. Закончить введение я бы хотел словами Аристотеля: "Математика выявляет порядок, симметрию и определённость, а это – важнейшие виды прекрасного". Мне кажется, что именно эти слова точно описывают раздел математики, который сейчас будет рассмотрен. Приступим.










