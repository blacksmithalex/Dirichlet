\chapter{Содержание первой главы}

В этой главе разберем краткие выводы, которые читатель может сделать из первой главы, приведем свойства, теоремы и части их доказательств. 

\section{Параграф \S 1}
В первом параграфе строго обосновываются переместительное и сочетательное свойство умножения. Автор указывает на то, что это необходимо, так как именно такие базисные свойства имеют важное значение в теории чисел. Выводы можно сформулировать следующим образом:

\begin{enumerate}
    \item При умножениии двух целых положительных чисел множимое можно заменить множителем и обратно; по этой причине исчезает различие между названиями ”множимое” и ”множитель”, и они оба называются сомножителями: $ab = ba$
    \item В случае трех целых положительных чисел порядок выполнения операции умножения неважен, то есть: $(ca)b = (cb)a = c(ab)$
\end{enumerate} 

\section{Параграф \S 2}
Параграф начинается с замечания о том, что утверждения \S 1 могут иметь место для любой системы S положительных чисел $a, b, c, ...$
Проводим вычисления следующим образом: берем какие-нибудь два числа, принадлежащие системе $S$, и составляем их произведение; система чисел $S'$, состоящая из этого произведения и остальных чисел системы S, содержит одним числом меньше, нежели система $S$.

\section{Параграф \S 3}

Вводится определения кратного числа. Когда число $a$ есть произведение двух чисел $b$ и $m$, то $a$ называется кратным b. Из этого определения вытекают следующие положения, которые далее будут часто использоваться, а именно:

\begin{enumerate}
    \item Если $a$ - кратное $b$, $b$ - кратное $c$, то $a$ - кратное $c$. То есть выполняется свойство транзитивности. Стоит заметить, что если в ряду чисел каждое делится на следующее за ним, то каждое число есть кратное всех последующих чисел.
    \item Если числа $a$ и $b$ - кратные числа $c$, то их сумма и разность тоже кратные числа $c$.
\end{enumerate}

\section{Параграф \S 4}

В учении о делимости чисел весьма важное значение имеет следующий вопрос: даны два целых положительных числа $a$ и $b$; нужно найти общие делители этих чисел, то есть такие числа $\delta$, которые делят одновременно $a$ и $b$.

В параграфе рассматривается деление с остатком и приводится алгоритм Евклида, который позволяет рекурентно искать наибольший общий делитель двух чисел. Из доказательства этого утверждения следует, что общие делители чисел $a$ и $b$ вполне совпадают с общими делителями некоторого числа $h$, которое всегда может быть найдено при помощи указанного алгоритма. Это число $h$ называется наибольшим общим делителем чисел $a$ и $b$.

\section{Параграф \S 5}

В этом параграфе особое внимание обращается на тот частный случай, когда НОД чисел $a$ и $b$ равен единице. Такие числа называют взаимно простыми. Если $a$ и $b$ - числа взаимно простые, а $k$ - произвольное число, то всякий общий делитель чисел $ak$ и $b$ есть в то же время общий делитель чисел $k$ и $b$.

Обращается внимание на частные случаи этой теоремы:

\begin{enumerate}
    \item Если $a$ и $k$ есть числа взаимно простые с $b$, то и произведение $ak$ есть число взаимно простые с $b$.
    \item Eсли $b$, будучи проcтым относительно $a$, делит произведение $ak$, то оно делит и $k$.
    \item Если имеем два ряда таких чисел $a,b,c,d,...$ и $\alpha, \beta, \gamma, \delta,...$, что каждое число первого ряда простое относительно каждого числа второго ряда, то произведение $abcd ...$ всех чисел первого ряда будет простым относительно произведения $\alpha\beta\gamma\delta...$ всех чисел второго ряда.
    \item Если числа $a$ и $\alpha$ взаимно простые, то всякая степень числа a должна быть взаимно простой относительно всякой степени числа $\alpha$.
\end{enumerate}

\section{Параграф \S 6}

Если дан ряд чисел $a,b,c,d,...$, то существует одно и только одно число $m$, имеющее то свойство, что всякое число, входящее множителем одновременно в $a, b, c, d, ...$, входит также и в $m$, и обратно, всякий делитель $m$ должен делить все данные числа $a,b,c,d,...$. Это число $m$ называется общим наибольшим делителем данных чисел. Исключение составляет тот случай, когда все данные числа равны нулю. Ряд чисел называется взаимнопростым, если и только если всякие два из них взаимно простые.

\section{Параграф \S 7}

В этом параграфе рассматривается вопрос обратный тому, который описан в предыдущем параграфе: дан ряд чисел $a,b,c,d,...$ и требуется найти все общие кратные данных чисел, т. е. такие числа, которые делятся на каждое из данных чисел. Нетрудно доказать, что все кратные чисел $a,b,c,d,...$ совпадают с кратным некоторого вполне определенного числа $\mu$, которое называется общим наименьшим кратным данных чисел.

Всякое число, которое делится на каждое из вазимно простых чисел $a,b,c,d,...$, делится также и на их произведение $abcd...$

\section{Параграф \S 8}

В этом параграфе формулируется теорема о единственности представления любого составного числа в виде произведения конечного числа простых чисел. Напомним, что составным является число, которое кроме единицы и самого себя делится еще на какое-то значение. 

Теорема формулируется следующим образом: всякое составное число всегда может быть прелставлено и притом только одним способом в виде произведения конечного числа простых чисел. Доказательство этой теоремы мы опустим и подметим следующее важное следствие: мы можем разложению составного числа $m$ дать более удобную форму в виде произведения простых чисел в степенях, то есть $m = a^{\alpha}b^{\beta}c^{\gamma}...$. 

Стоит отметить, что эта теорема имеет большой вес для решения ряда задач, а также доказательства некоторых утрвеждений. К примеру, зная разложение составного числа в произведение простых в степенях, можно определить количество всех делителей числа $m$ по формуле $(\alpha + 1)(\beta + 1)(\gamma + 1)...$ Это следует из того, что любой делитель числа $m$ есть некоторая комбинация произведения простых чисел, входящих в разложение числа $m$, в степенях от 0 (что значит, что мы не берем данный делитель) до максимальной.

\section{Параграф \S 9}

Только что доказанная теорема дает весьма удобный критерий для решения вопроса, делится ли число $m$ на число $n$, если мы допустим только, что оба числа уже разложены на простые множители. Критерий основан на идеях, которые описаны в предыдущем параграфе, поэтому мы не будем акцентировать на нем внимание. Обоснование можно также найти в книге.

\section{Параграф \S 10}

В этом параграфе мы возвращаемся к задаче, которая была рассмотрена в 6 параграфе. Она заключается в нахождении общего наибольшего делителя нескольких чисел, предполагая, что эти числа разложены на простые множители. 

Из этих последних мы иключаем все те, которые не входят в одно или в несколько данных чисел. Если бы таким обраом пришлось исключить все простые множители, тогда общим наибольшим делителем данных чисел являлась бы единица. В противном случае, остается после этого предварительного исключения, например число, которое входит по крайней мере один раз в разложение каждого из данных чисел. 

Затем считаем сколько раз это число входит в разложение каждого из этих чисел и получаем значение показателя для данного числа. Подобным образом поступаем со всеми остальными числами и находим значение показателей для них. Этот способ нахождения наибольшего делителя основан на том, что общий наибольший делитель может содержать только те простые множители, которые входят в каждое из данных чисел, и притом каждый их этих множителей входит в общий делитель не чаще, чем каждое из данных чисел. Абсолютно аналогичным образом решается задача о нахождении наибольшего общего кратного нескольких чисел. В состав наименьшего общего кратного должны входить все простые числа, входящие в состав данных чисел, и при том с наибольшими показателями.

\section{Параграф \S 11}

В этом параграфе нас знакомят с так называемой функцией Эйлера $\varphi(m)$, которая возвращает количество чисел от $1$ до $m$ взаимнопростых с $m$. В параграфе ставится вопрос о нахождении общего выражения для функции $\varphi(m)$. То есть необходимо решить следующую задачу:

Числа $a,b,c ...$ являются взаимно простыми и входят в качестве множителей в $m$. Требуется определить сколько чисел в ряде $1, 2, 3, ..., m$ не будут делиться ни на одно из чисел $a,b,c ...$

Опишем кратко алгоритм решения данной задачи. В первую очередь необходимо исключить из ряда, который мы описали выше, все числа, которые делятся на число $a$: это числа $a, 2a, 3a,..., {m \over a}a$. Их число равно $m \over a$. Если мы исключаем эти числа из ряда, то по понятным причинам остается $m - {m \over a} = m(1 - {1 \over a})$ чисел, которые уже не делятся на $a$. Далее производим аналогичные действия для $b, c, ...$

Итого получим следующий вывод: eсли $a, b, c, ..., k, l$ - различные простые числа, которые входят в состав $m$, то количетсво чисел, взаимно простых с $m$ и входящих в ряд $1, 2, ..., m$ определяется формулой:

\begin{center}
    $\displaystyle m(1 - {1 \over a})(1 - {1 \over b})...(1 - {1 \over k})(1 - {1 \over l})$
\end{center}

\section{Параграф \S 12}

Из утверждения, доказанного в предыдущем параграфе, формулируется следующая теорема: Если $m$ и $m'$ взаимно простые, то $\varphi(mm') = \varphi(m)\varphi(m')$. Это свойство называется мультипликативным свойствоv функции и несет в себе важное значение для ряда вычислительных задач. Также стоит отметить, что теорема верна для произведения любого числа множителей в случае, если они попарно взаимно просты.

\section{Параграф \S 13}

Вопрос об определении функции $\varphi(m)$ представляет собой частный случай следующей задачи: дан ряд чисел $1,2,3,...,m$ и требуется найти число тех чисел этого ряда, которые
имеют общим наибольшим делителем с $m$ число $\delta$, причем $\delta$ есть один из делителей числа $m = n\delta$. Опишем доказательство, так как оно компактно и хорошо показывает применение предыдущих умозаключений.

Сведем эту задачу к предыдущей: очевидно, что все искомые числа находятся между числами $\delta,2\delta,3\delta,...,n\delta$. Для того, чтобы $\delta$ было наибольшим общим делителем числа $m = n\delta$ и какого-нибудь числа вида $r\delta$, необходимо и достаточно, чтобы $r$ и $n$ были числами взаимно простыми. Следовательно, искомых чисел столько, сколько найдется чисел ряда $1, 2, 3, . . . , n$, взаимно простых с $n$. Число таких чисел равно $\varphi(n)$. Когда $\delta = 1$, приходим к частному случаю, рассмотренному в предыдущем параграфе. Закончим тем, что приведем еще одно свойство функции $\varphi(m)$. Если $n$ пробегает значения всех делителей числа $m$, то имеет место следующее равенство:
\begin{center}
    $\sum_{n} \varphi(n) = m$
\end{center}

\section{Параграф \S 14}
Приведенное выше доказательство этого важного свойства функции $\varphi(m)$ получается непосредственно из определения и не требует предварительного определения вида самой функции. Дополнительно отмечу, что в в этом параграфе рассматривается другое доказательство этого же факта при помощи уже исвестного нам выражения для функции $\varphi(m)$. Не будем долго на этом останавливаться.

\section{Параграф \S 15}

Этот параграф посвящен изучению проблемы определения показателя наивысшей степени простого числа $p$, входящей множителем в факториал $m!$, где m - некоторое целое число. Пусть $m'$ - наибольшее целое число, содержащееся в дроби $m \over p$.  Тогда из числа $m$ сомножителей произведения $m!$ на $p$ делятся следующие $m'$ сомножителей: $p, 2p, ...,m'p$. Искомый показатель равен показателю наивысшей степени числа $p$, входящей в произведение $1*2*3*...*m'*p^{m'}$. Он равен сумме $m'$ и показателя наивысшей степени $p$, входящей в произведение $m′!$. Тогда имеем, что искомый показатель равен $m' + m'' + m'''+...$, где $m'', m''',...$ - наибольшие целые числа, которые содержатся в дробях $m' \over p$, $m'' \over p$ и т. д.

Анализируя информацию выше, можно прийти к следующей теореме: eсли $m = f + g + h + ...$, то отношение $m! \over f!*g!*h!...$ является целым числом. Эту теорему также оставим без доказательства.

\section{Параграф \S 16}

В этом параграфе автор делает подытог. Он обращает внимание на то, что большая часть описанных утверждений и теорем строится на основе алгоритма нахождения наибольшего общего делителя. В параграфе приводится пример целых комплексных чисел и показывается, как теория, описанная в этой главе для обыкновенных целых чисел, может быть расширена на целые комплексные. 