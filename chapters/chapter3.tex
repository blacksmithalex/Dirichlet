\chapter{Аннотация}

Книга П.Г. Лежена Дирихле ”Лекции по теории чисел” была составлена его другом, Рихардом Дедекиндом, по лекциям, прочитанным Дирихле в 1856-1857 годах в Гёттингенском университете. Сейчас эта книга считается одним из классических трудов по теории чисел, и она до сих пор не потеряла своей значимости в этой области математики.

Стоит отметить, что на немецком языке книга выдержала четыре издания, на русском языке появилась впервые в 1936 году. Книга состоит из пяти основных глав, а также 10 дополнительных глав: дополнений Дедекинда и приложения - статьи Б.Н. Делоне о геометрии бинарных квадратичных форм. В книге Дирихле дано чисто арифметическое изложение теории квадратичных форм, поэтому дополнительно необходимо изучить статью Делоне, чтобы более глубоко понять смысл важнейших теорем, относящихся к теории квадратичных форм.

Первые три главы книги состоят из теории, которая является классической для элементарных курсов по теории чисел. В этих главах излагаются общие свойства делимости чисел, общая теория сравнений и теория квадратичных вычетов. Две последние главы содержат подробное изложение арифметической теории квадратичных форм. 

В этом реферате я хочу более подробно разобрать теорию, которая изложена в первой главе, так как она является основополагающей для изучения последующей теории.
