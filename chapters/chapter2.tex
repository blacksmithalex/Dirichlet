\chapter{Биография ученого}

Иоганн Петер Густав Лежён-Дирихле (нем. Johann Peter Gustav Lejeune Dirichlet) - немецкий математик, внёсший существенный вклад в математический анализ, теорию функций и теорию чисел. Член Берлинской, Петербургской и многих других академий наук.

Дирихле родился 13 февраля 1805г. в вестфальском городе Дюрене в семье почтмейстера. Его предки были выходцами из бельгийского городка Ришле (Richelet), этим обусловлено происхождение необычной для немецкого языка фамилии. Часть фамилии «Лежён» имеет аналогичное происхождение - деда называли «молодым человеком из Ришле» (фр. Le Jeune de Richelet).

\section{Дество и юность}

Дирихле с малых лет увлекался математикой и тратил деньги, которые ему давали на карманные расходны, на покупку учебников по этой науке. В 12 лет он поступил в гимназию в Бонне, в 14 лет - в иезуитский колледж в Кёльне, где он учился у Георга Ома (1789-1854 гг.). К 16 годам Дирихле получил школьное образование и был готов поступать в университет. Он отправился во Францию, привезя с собой работу немецкого математика Гаусса (1777- 1855 гг.). В качестве учителей Дирихле были некоторые из ведущих математиков, и он смог извлечь большую пользу из опыта общения с такими французскими математиками как Фурье (1768-1830 гг.), Франкёром (1773-1849 гг.), Лапласом (1749-1827 гг.), Лакруа (1765- 1843 гг.), Лежандром (1752-1833 гг.) и Пуассоном (1781-1840 гг.). С лета 1823 года Дирихле преподавал немецкий язык жене и детям генерала Максимильена Себастьена Фуа (1775- 1825 гг.), живя в его доме в Париже. В это время начинают появляться первые научные работы Дирихле.

\section{Статьи, открытия, достижения}

В 1825 году Дирихле вместе с Лежандром доказал великую теорему Ферма для частного случая $n = 5$. Теорема утверждала, что для $n > 2$ не существует ненулевых целых чисел $x,y,z$ таких, что $x^n$ + $y^n$ = $z^n$. Случаи $n = 3$ и $n = 4$ были доказаны Эйлером (1707-1783 гг.) и Ферма (1601-1665 гг.). 

Доказательство разбивалось на 2 случая. Дирихле доказал случай, когда одно из чисел $x,y,z$ делится на $5$ и на $2$, и представил свою работу Парижской академии в июле 1825 года. Лежандр был назначен одним из рецензентов, и он смог доказать случай, когда одно из чисел $x, y, z$ делится на $5$, а другое из $x, y, z$ делится на 2, завершив доказательство. Полное доказательство было опубликовано в сентябре 1825 года. 

После этого Дирихле дал доказательство теоремы Гаусса для биквадратичных остатков. Помимо этого он показал большую роль анализа и теории аналитических функций для решения проблем теории чисел. Известна доказанная им теорема о существовании бесконечно большого числа простых чисел во всякой бесконечной арифметической прогрессии из целых чисел, первый член и разность которой - числа взаимно простые. Дирихле первым дал точное доказательство сходимости рядов Фурье, известное как признак Дирихле, а в вариационном исчислении привел так называемый принцип Дирихле.

В 1827 году юноша по приглашению Александра фон Гумбольдта устроился на должность приват-доцента университета Бреслау (Вроцлав). В 1829 году он перебирается в Берлинский университет, где проработал непрерывно 26 лет, сначала как доцент, затем (с 1831 года) как экстраординарный, а с 1839 года как ординарный профессор Берлинского университета.

В 1855 году Дирихле становится в качестве преемника Гаусса профессором высшей математики в Гёттингенском университете. 